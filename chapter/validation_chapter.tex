\chapter{离子电流的工况拓展}
在上一章中主要探讨的是离子电流未被淹没情况下的离子电流信号分析手段,主要包括小波分析去除干扰,高斯曲线拟合提取离子电流特征参数。
这些手段同样也可以用于淹没情况下的离子电流信号处理,且由于淹没情况下的离子电流信号无法直接提取特征参数,利用上述方法对离子电流信号
进行处理成为了非常有效的手段。本章就这些方法对离子电流工况拓展的可行性进行验证。
\section{离子电流基本特征参数}
离子电流基本特征参数主要包括,火焰前锋期时间$t_c$,火焰前锋期峰值相位$d_c$,火焰后期时间$t_r$,火焰后期峰值相位$d_r$以及离子电流积分值$I_i$等。火焰前锋期和火焰后期的时间
和整个缸内混合气的燃烧过程有关,所以和滞燃期、急燃期和后燃期有相关性。通常我们采用放热率的百分比来定义这几个燃烧的时期。同时根据文献可以知道,火焰后期峰值相位
和缸压峰值相位之间有稳定的线性关系,而火焰前锋期峰值相位应该也和缸压的特定点具有稳定的相位关系。\par
除此之外,根据两个放电干扰可以确定点火的确切时间,从而可以判定滞燃期的具体时间长短。
\section{离子电流工况拓展方法的验证}
主要的依据是火焰后期峰值相位和缸压峰值相位的线性关系,淹没情况下的火焰后期峰值大小和未被淹没下的离子电流火焰后期峰值大小直接的一致相关性。
\section{离子电流火焰前锋期的特性}
离子电流火焰前锋期很容易被点火干扰淹没,所以离子电流火焰前锋期的特征参数在较多工况下都是无法获得的。
\section{离子电流积分值的特性}
纯电火干扰的存在影响了离子电流积分值的计算