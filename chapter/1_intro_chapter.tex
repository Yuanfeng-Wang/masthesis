\chapter{绪论}
\section{前言}
早在十九世纪,人们就观察到碳氢化合物在燃烧时会产生自由离子,使得火焰和燃烧产物具有导电能力。这一现象被注意到,
并逐渐运用到燃烧过程研究工作。由于火花塞点火,发动机上止点附近的燃烧产生大量的离子在缸内无规则的移动,且随着缸内燃烧温度的逐渐增加,
缸内离子的运动越加的剧烈。这种剧烈运动的离子如果收到火花塞两极之间加上的直流偏置电压,就会形成离子电流。这种电流能够很好地
对缸内的燃烧状况进行指示\cite{xjgxcs}。\par
但是,由于以往对发动机控制技术要求不高,同时受发动机控制软、硬件系统的整体技术水平限制,该方法并未引起人们的广泛
注意。近年来,随着对发动机效率及排放性能要求的逐渐提高,该方法逐渐引起重视。离子电流检测技术的最
初应用主要是面向于失火现象的检测,主要原因是当时的法规要求在所有负荷和转速下实现100\%失火检测\cite{grzys}。而传统的失火检测方法并不能够完美地检测每一次失火,尤其是基于转速的曲轴转速波动法,
在颠簸的道路上行驶效果非常的差。由此,对离子电流检测技术的研究逐渐由对燃烧过程中的失火、爆震等现象的定性检测,逐渐向空燃比、燃烧相位及缸内燃烧压力状态等参数的定量检测上来。
\section{离子电流国外研究现状}
国外对于离子电流的研究较为成熟,对离子电流检测技术的研究主要集中于离子电流的形成机理、特征提取及参数估计方法、检测与应用等方面。
早在1953年Winch和Mayes\cite{winch1953method}就利用了偏置电压式的离子电流检测电路记录了离子电流信号,并以此分析了缸内燃烧状况。在20世纪
50年代,出现了大量的研究火焰中化学离子平衡和火焰后燃区的研究成果。进入80年代,汽车领域的科技工作者致力于离子电流信号的研究,取
得了很多重要的成果。1986至1997年Nick Collings\cite{collings1986knock,collings1991plug,collings1985knock}发表了利用火花塞作为传感器检测爆震的三篇论文,介绍了离子电流
法的基本原理和离子电流的发展历程,并完善了爆震检测的方法。1996年Andre Saitzkoff\cite{saitzkoff1996ionization}发表了火花塞离子电流的平衡计算的论
文,利用偏置电源式离子电流检测电路测量了离子电流信号,通过分析得到了火花塞离子电流的近似理论计算公式。\par
在离子电流的形成机理研究方面, Saitzkoff和Reinmann等人\cite{saitzkoff1996ionization}率先对$N_{2}$,$H_{2}O$,$CHO$和$NO$四种基团在离子电流形成过程中所起的作用
进行了对比分析。研究认为燃烧火焰电离后能够定向移动的自由离子只占极小部分,其原因在于移动能力上的差异。同时,研究表明占离子总浓
度1%的$NO$基团产生离子电流的能力最强。Kessler等人\cite{kessler}在柴油机上采用光学方法对离子电流的研究表明,由于柴油和汽油在燃烧时产
生自由离子的化学反应过程不同,二者产生的离子电流信号波形差异显著。在汽油机上形成的离子电流中,电子是负电荷的主要载体,而对于
柴油机上的所形成的离子电流,带负电的离子也需考虑。A.Franke\cite{franke2003analysis}在对汽油机离子电流产生的物质来源的分析基础上,首次将离子
电流的产生划分为火焰前锋期和火焰后期两个阶段,认为在火焰前峰期的离子电流主要是由火焰前锋面与火花塞电极接触时所产生的,而
在火焰后期热电离过程是形成离子电流的主要原因。\par
除了离子电流产生的化学动力学机理外,气缸结构及火花塞位置等物理因素也是影响离子电流波形的重要成因。L.Peron等人\cite{peron2000limitations}对离子电流
信号的局限性分析认为,离子电流生成机理并不是决定信号波形的唯一原因。
在基于离子电流的参数估计方法方面,由于离子电流信号的产生机理复杂,信号波形变化规律多样,在利用该信号进行燃烧特征检测时
,大多采用了数学模型比如神经网络的方法,来建立电流信号特征与被估参数之间的映射关系。\par
Gerard等人\cite{malaczynski2003real}针对不同燃烧边界条件下,离子电流信号变化规律复杂的情况,提出了基于信号时域特征,采用主成分分析方
法对离子电流信号进行了特征提取,该方法可以有效的对离子电流信号观测窗内观测样本进行降维,从而减小后续的参数估计算法的计
算量。Hellring\cite{hellring1999robust}采用神经网络的方法成功地对空燃比进行了估计,并且在试验车中进行实验验证。
Hellring\cite{hellring1999spark}采用神经网络的方法成功地对点火提前角进行了控制,且试验车在高速中进行了验证。
目前对离子电流的应用已经从初期对失火及爆震的定性研究转入对燃烧参数的定量研究。\par
在基于离子电流的爆震研究方面,J.Forster等人\cite{forster1999ion}率先基于离子电流的频域特征,实现了对爆震及其强度的准确估计。该研
究展示了爆震强度与离子电流频域信号高频区域内幅值的相关性。此后,一系列针对产品发动机爆震检测的相关研究也相继出现。这些研究
中普遍采用了离子电流的频域特征实现了较高质量的爆震检测。\par
在空燃比检测方面,Reinmann\cite{reinmann1997local}利用离子电
流公式理论和实验论证拟合了空燃比-离子电流公式。Kenneth Ratton和Ming C. Lai等人\cite{balles1998cylinder}利用火花间隙电离检测研究了缸
内空燃比的近似值,观察到了空燃比和离子电流特征之间的较强相互关系。其研究认为,从整体性能上看,离子电流检测无法取代缸压检测,二者
信号质量和性质都完全不同。但通过具体工况下对离子电流波形的具体分析,该方法仍具备有对空燃比进行实时反馈的潜力。\par
在对缸内燃烧压力状态的检测方面,Chao.F等人\cite{zhu2003mbt}提出,部分负荷工况下,对离子电流信号进行了多循环平均处理,通过处理后信号的第二峰位置
及幅值,实现了对缸内燃烧峰值压力及其位置的准确预测。\par
除此之外,2000年Wayne University的D. Schneider\cite{schneider2000experimental}发表了离子电流与内燃机工作参数之间关系的博士论文。
Magnus等人\cite{glavmo1999closed}在研究柴油机EGR和颗粒污染对离子电流的影响时,尝试了在稳态工况下,通过多循环均值后的离子电流信特征值
号对燃烧始点进行反馈和控制。先预定一个离子电流信号强度阈值,当检测到的离子电流信号首次越过该阈值时,该时刻被定义为燃烧
始点。对燃烧始点检测成功后,可以通过调整燃油十六烷值和空气、燃油以及发动机温度等相应方法,对滞燃期加以调整。
\section{离子电流国内研究现状}
1998年西安交通大学的吴筱敏发表了关于离子电流检测爆震的文章\cite{wxm1998wxm},并于次年详细介绍了离子电流
的检测原理\cite{wxm1999wxm},分析了信号电压与偏置电源电压、火花塞电极间隙之间的关系。这两篇文章是国内早期为数不多的详细阐述离子电流法的文献。
随后的国内离子电流研究主要着重于应用层面的使用,比如失火检测、爆震检测和早燃检测等,主要是对于国外的技术手段改进升级。\par
汪映\cite{wy2002}采用偏置电源式离子电流检测电路对失火和爆震进行了检测,白华\cite{bh2007}对汽油机的燃烧状态进行了动态监测。董光宇\cite{dgy2008ys}基于
离子电流技术对发动机指示转矩进行了实时估计,且发现离子电流经过傅里叶变换后的幅值与转矩对应关系更明显。刘兵\cite{lb2015lcy}在定容燃烧弹上采用了离子电流法测量了混合气燃烧
火焰传播速度,其相对误差小于7\%。高忠权\cite{gzq2015lcy}在点火式发动机中不同位置布置了多个离子电流测量电极,分析了离子电流与工况参数之间的关系,其结果表明
多离子电流测量电极布置方式可以有效地提高离子电流信号的信噪比。郑兵艳\cite{zby2015}采用电容式离子电流检测电路研究了离子电流的循环变动系数和燃烧相关性。\par
虽然国内的离子电流的研究较晚,但是关于HCCI燃烧离子电流的研究,国内学者造诣颇深。2004年天津大学的李建文\cite{ljw2004phd}在其博士论文中
详细并且全面地阐述了离子电流的形成机理和数学模型,并对HCCI燃烧离子电流进行了研究,是国内较早对HCCI燃烧离子电流进行研究的文献。
但是在HCCI燃烧离子电流的研究方面,同济大学李理光课题组\cite{dgy2011,zqy2011,fqw2012,zzybh2012,cyb2013}
较为全面地对于HCCI燃烧离子电流特性进行了研究。\par
在HCCI燃烧离子电流的化学机理研究方面,张志永\cite{zzy2011hcci}利用化学反应动力学方法,建立了耦合离子反应机理的异辛烷HCCI燃烧模型,研究了HCCI燃烧离子电流形成过程中的主要
离子和中间自由基的来源及生成历程,并与仿真的结果进行了对比。经过研究发现,最大燃烧放热率点之前主要以$CHO^{+}$为主要离子,之后由于电子中和效果,$H_{3}O^{+}$成为了
离子电流的主要离子。\par
在HCCI燃烧离子电流特征研究方面,李从跃\cite{lcy2010hcci}开发了HCCI汽油机实验台架,桑文\cite{sangwen2010hcci}在该实验台上
基于HCCI燃烧方式,通过改变燃烧边界条件,对离子电流信号的特性进行了研究。
经过研究发现,进气温度、混合气浓度、压缩比和发动机转速是HCCI燃烧离子电流的主要影响因素,HCCI燃烧离子电流的相位与燃烧相位之间有一定的相位延迟。
董光宇\cite{dgy2011}在缸内直喷汽油机上,基于离子电流的HCCI燃烧检测方法进行了研究。研究发现,过量空气系数对信号特征值与燃烧相位的相关性的影响较大,过量空气系数大于1.6之后两者
的相关性大幅降低;随着转速的提高,通过离子电流对燃烧相位进行检测的准确性提高。曹银波研究了喷油策略对HCCI燃烧离子电流的影响,发现喷油提前相位与离子电流特征值相位成正相关性;
喷油时刻对离子电流与燃烧相位的相关性影响不大;首次喷油较于二次喷油对离子电流与燃烧相位的相关性影响更大。康哲\cite{kz2014wzj}分析了不同工况参数对柴油机离子电流与燃烧相位相关性的影响。
结果表明,在不同工况参数下离子电流信号特征值与燃烧相位相关系数均接近或达到0.8,且受燃烧循环变动影响。\par
在HCCI燃烧离子电流的应用方面,张栖玉\cite{zqy2011}在HCCI汽油机实验台架上分析了不同失火工况下的离子电流积分信号特征及其与燃烧的相关性,并基于离子电流积分反馈的循环失火信号进行了补火
闭环控制的试验。结果表明,离子电流积分值信号可以检测失火,与国外对于非HCCI燃烧离子电流检测失火的研究结果一致。张志永\cite{zzy2009nb}研究了不同的内部EGR率对自由离子生成和HCCI燃烧过程诸参数
的影响,研究表明$CHO^+$的峰值摩尔分数随着EGR率的增加呈现先增加后减少的趋势。曹银波\cite{cyb2012}对离子电流检测系统进行了抗干扰设计并提出电容检测法无法准确反映发动机的爆震现象,离子电流的相位延迟
问题可以用算法进行校正。刘寅童\cite{lyt2013dj}利用缸内直喷HCCI发动机台架,对乙醇HCCI燃烧的瞬态空燃比进行了研究。结果表明,喷油量的变化,发动机转速和
进气温度对瞬态空燃比有较大影响。\par
\section{高斯拟合研究现状}
1996年Andre Saitzkoff\cite{saitzkoff1996ionization}发表了火花塞离子电流的平衡计算的论
文,利用偏置电源式离子电流检测电路测量了离子电流信号,通过分析得到了火花塞离子电流的近似理论计算公式。该离子电流的近似理论计算公式近似于高斯函数,由此国内外学者
开始使用高斯函数来拟合离子电流曲线,从而计算得到离子电流的特征值。\par
1997年L.Eriksson和L.Nielsen\cite{eriksson1996ignition,eriksson1997closed,eriksson1997ionization}采用双高斯曲线拟合方法准确地估计缸压最值对应相位,
自适应地调整点火提前角。Magnus Hellring\cite{hellring2001comparison}等人也通过了拟合曲线的方法控制了缸压峰值相位的位置。
Henrik Kl\"{o}vmark\cite{klovmark2000estimating}采用了高斯曲线拟合的方式有效地对空燃比进行了估计,且在中速高负荷工况下可以达到0.1\%的误差范围。
Gazis等人\cite{panousakis2006ion}通过双高斯曲线拟合的方法,对电流信号的时域特征进行了提取,并针对离子电流信号的非线性变化特征,采用
神经网络方法实现了对汽油机燃烧峰值压力位置的准确预测。\par
国内的魏若男\cite{wrn2013}采用高斯拟合法对离子电流信号曲线进行拟合,采用BP神经网络算法根据离子电流特征参数计算了缸内压力参数,其平均绝对误差小于$1\si{MPa}$,平均相对
误差小于2\%。
\section{研究目的、意义及内容}
\subsection{研究目的及意义}
常见的离子电流检测电路分为电容式和偏置电源式。偏置电源式检测电路可以将点火干扰屏蔽,获得的离子电流曲线便于分析计算\cite{cyb2012},但
是由于需要使用高压硅堆和外置电源成本太高,不利于大批量的商业化使用。电容式检测电路结构简单,成本低廉,但是点火干扰阶段容易和火焰前锋期甚至包括焰后期重
叠,导致离子电流的一些特征参数无法识别。如果能够提出一种较好的信号处理方法处理点火干扰阶段信号,将会使成本低廉、结构简单的电容
式检测电路得到推广。\par
在传统的离子电流信号曲线的特征值提取方法中,特征参数的提取有限,且往往由于缸内循环不稳定导致特征参数的提取有一定的误差。除此之外,特征参数的计算往往
会消耗发动机控制系统中的计算资源,如果能够提高特征值计算算法的效率,将会大大提高发动机控制系统的可靠性并可能降低发动机控制系统的硬件成本。
\subsection{主要研究内容}
从目前国内外研究现状来看,离子电流的基础理论研究比较健全,基本特征和燃烧之间的关系的研究也较多,但是大多数采用的是偏置电源式的离子电流检测电路进行实验。对于电容式
离子电流检测电路的研究较少。本文详细分析了电容式离子电流检测电路中点火干扰的来源,采用小波分析的方法去除点火干扰,并通过双高斯曲线拟合算法快速计算了离子电流中的
各特征参数,提高了计算效率,拓展了离子电流适用的工况。主要工作如下:
\begin{enumerate}[(1)]
\item 搭建了发动机实验台架,在该实验台架上使用电容式离子电流检测电路采集离子电流信号,获取了后续建立双高斯曲线拟合模型的离子电流曲线;
\item 分析了电容式离子电流检测电路中的点火干扰的来源,分析了点火干扰的特性,并且采用同一工况下的断油和断火循环计算得到了纯点火干扰信号;
\item 确定了小波分析方法的小波基函数和阶数的选择,采用小波分析方法去除了离子电流信号中的点火干扰;
\item 开发了双高斯曲线拟合算法,结合双高斯曲线参数快速计算了离子电流的各特征参数值。
\item 结合发动机工况参数验证了双高斯曲线拟合模型的正确性;通过该拟合模型计算得到了工况参数的拟合估计值并计算其循环波动系数,计算了拟合估计值和
工况参数之间的相关系数。
\item 计算了拟合估计值的循环波动系数和相关性随转速变化情况。
\end{enumerate}
