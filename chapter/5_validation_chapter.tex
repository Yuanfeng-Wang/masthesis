\chapter{离子电流的适用工况拓展}
在上一章中主要探讨的是离子电流未被淹没情况下的离子电流信号分析手段,主要包括小波分析去除干扰,高斯曲线拟合提取离子电流特征参数。
这些手段同样也可以用于淹没情况下的离子电流信号处理,且由于淹没情况下的离子电流信号无法直接提取特征参数,利用上述方法对离子电流信号
进行处理成为了非常有效的手段。本章就这些方法对离子电流工况拓展的可行性进行验证。
\section{离子电流基本特征参数}
离子电流基本特征参数主要包括,火焰前锋期时间$t_c$,火焰前锋期峰值相位$d_c$,火焰后期时间$t_r$,火焰后期峰值相位$d_r$,离子电流积分值$I_i$和离子电流升高率$\phi$等。\par
火焰前锋期与火焰经过电极两段时的火焰前锋面有关,由于时间比较短,而且没有深刻的理论基础,无法很好的得到火焰前锋期的具体形状,从而很难判断火焰前锋期的时间和
峰值相位。西安交通大学的吴筱敏\cite{iign}分析了不同电极间距下的离子电流信号曲线,如图\label{fig:intersection_ign}所示的是改变电极间距$\delta$测量得到的离子电流曲线。
\begin{figure}[!htpb]
	\centering
	\includegraphics[width=0.65\textwidth]{thesis_figure/ion_chapter/intersection_ign}
	\caption{\label{fig:intersection_ign}不同电极间距的离子电流信号曲线}
\end{figure}
从图\label{fig:intersection_ign}中可以看到火焰前锋期只有一个很小的波峰,且随着电极间距离的增加火焰前锋期越加的不明显。若要通过算法计算火焰前锋期很难实现。\par
火焰后期与整个缸内混合气的燃烧过程有关,且根据Andre Saitzkoff\cite{saitzkoff1996ionization}的理论可以知道,火焰后期主要和NO的热电离有关系,其形状近似高斯函数。
因此可以推断火焰后期和滞燃期$t_{\theta_1}$、急燃期$t_{\theta_2}$和后燃期$t_{\theta_3}$有相关性,通常我们采用放热率CA的百分比来定义这几个燃烧的时期,如下公式所示:
\begin{align}
	t_{\theta_1}^{'} &= CA0  & t_{\theta_1}^{''} &=CA10 \\
	t_{\theta_2}^{'} &= CA10 & t_{\theta_2}^{''} &=CA90 \\
	t_{\theta_3}^{'} &= CA90 & t_{\theta_3}^{''} &=CA100 
\end{align}
而根据双高斯曲线拟合方法,可以定义高斯曲线的宽度的两倍即为火焰后期宽度。从而可以根据高斯
曲线的中心位置和宽度计算出火焰后期的开始时刻$t_{r}^{'}$和结束时刻$t_{r}^{''}$。由此可以分析火焰后期和滞燃期$t_{\theta_1}$、急燃期$t_{\theta_2}$、后
燃期$t_{\theta_3}$三者之间的关系。\par
离子电流的积分值$I$的计算方式采用去除纯压缩导致的热电离的离子电流积分方式,NO热电离是导致离子电流的主要原因,因此产生NO过程的燃烧对应的做功和离子电流积分值会有对应的关系。\par
最大离子电流升高率$\phi$的计算方法采用的是用拟合高斯曲线的高度和拟合高斯曲线的宽度的比值。最大离子电流升高率对应相位$d_{\phi}$的计算方法采用的是拟合高斯曲线的中心位置
向左移动半个宽度对应的相位。\par
根据以上的总结可以得到离子电流基本特性参数的计算公式如下:
\begin{align}
	t_{r}^{'} &= g_{c}-g_{w}\\
	t_{r}^{''} &= g_{c}+g_{w}\\
	d_{r} &= g_{c}\\
	d_{\phi} &= g_{c}-\frac{g_w}{2}\\
	\phi &= \frac{g_{h}}{g_{w}}\\
	I &= g_{h}e^{-(\frac{g_c}{g_w})^2}+g_{h}^{'}e^{-(\frac{{g_c}^{'}}{{g_w}^{'}})^2}
\end{align}
其中$g_{c}$,$g_w$,$g_h$分别是火焰后期即第二拟合高斯函数的中心曲轴转角,宽度和高度。$g_{c}^{'}$,$g_w^{'}$,$g_h^{'}$是第一高斯函数的中心曲轴转角,宽度和高度。
其中积分值$I$的计算方法采用的是拟合高斯函数和标准高斯函数的比值来计算,因为标准高斯函数的积分值为1,这样可以大大缩减计算时间。
\section{离子电流工况拓展方法的验证}
采用上一章节中的小波分析方法和双高斯曲线拟合方法来分析离子电流淹没工况下($3000r/min$)的单个随机循环的离子电流,得到如图\ref{fig:5_1_merged_ion}所示的结果。
\begin{figure}[htb]
	\centering
	\includegraphics[width=0.9\textwidth,trim=1cm 0.75cm 1cm 0.75cm,clip]{thesis_figure/ion_chapter/5_1_merged_ion}
	\caption{\label{fig:5_1_merged_ion}淹没工况下离子电流的小波分析处理}
\end{figure}
从图\ref{fig:5_1_merged_ion}中可以看到火焰后期的尖峰被分离了出来。从单个循环的角度来看,该离子电流工况拓展方法是有效的,还需要经过多循环分析的讨论。多循环分析主要从两个基本原理和已经
被证实的离子电流特性共三个角度来考虑。\par
第一,从原理的角度上来看,对离子电流信号进行小波分解,将震荡信号消除不会将真实的离子电流信号去除。因为离子电流产生的基本原理是离子的移动,不具有高频往复的特性,不会产生震荡信号。震荡信号的主要
来源是点火线圈的震荡。因此小波分析去除干扰的方法有很强的科学性。\par
第二,从原理的角度上来看,火焰后期主要和NO的热电离有关系,其形状近似高斯函数。火焰后期和滞燃期$t_{\theta_1}$、急燃期$t_{\theta_2}$和后燃期$t_{\theta_3}$有相关性。CA10表征的是滞燃期的开始时刻,而CA90表征
火焰后期的结束时刻。因此CA10和CA90表示的是整个燃烧的开始和结束。考虑离子电流淹没工况下($3000r/min$)通过计算随机连续20个循环的CA10、CA90和火焰后期开始时刻和结束时刻得到
如图\ref{fig:5_1_ca_m2center_start_end}所示的曲线。\par
\begin{figure}[htb]
	\centering
	\includegraphics[width=0.8\textwidth]{thesis_figure/ion_chapter/5_1_ca_m2center_start_end}
	\caption{\label{fig:5_1_ca_m2center_start_end}CA10、CA90和火焰后期开始时刻和结束时刻}
\end{figure}
从图中可以看到,CA10和火焰后期开始时刻比较吻合,而CA90和火焰后期结束时刻比较吻合。这符合火焰后期主要是燃烧导致的NO电离导致的。\par
\begin{figure}[htb]
	\centering
	\includegraphics[width=0.8\textwidth]{thesis_figure/ion_chapter/5_1_pph_m2center}
	\caption{\label{fig:5_1_pph_m2center}缸压峰值相位和火焰后期峰值相位}
\end{figure}
第三,根据文献可以知道,火焰后期峰值相位和缸压峰值相位之间有稳定的线性关系,而火焰前锋期峰值相位应该也和缸压的特定点具有稳定的相位关系。从图\ref{fig:5_1_merged_ion}中我们也可以
看到火焰后期的峰值相位和缸压峰值相位之间确实非常的接近。考虑离子电流淹没工况下($3000r/min$)通过计算随机选取的连续20个循环的缸压峰值相位和火焰后期峰值相位图,
可以得到如图\ref{fig:5_1_pph_m2center}所示的曲线。
从图\ref{fig:5_1_pph_m2center}中可以看到两者之间有很强的线性关系,且各个循环两者的差值在5度左右,总体的变化趋势保持一致。\par
总结以上三个方面的推导,理论上NO热电离产生的离子电流没有高频信号,经过小波分析得到的离子电流信号的峰值相位和缸压峰值相位有很好的对应关系,对小波分析得到的离子电流信号进行双高斯曲线拟合
计算得到的火焰后期的初始时刻和CA10对应,火焰后期的结束时刻和CA90对应,由此我们可以判定采用小波分析和双高斯曲线拟合方法可以科学准确地计算得到离子电流的特征参数,该离子电流工况拓展的方法具有
科学性和有效性。
\section{高斯拟合曲线计算特征参数}
\subsection{离子电流火焰前锋期的不确定性}
离子电流火焰前锋期很容易被点火干扰和噪声淹没,所以离子电流火焰前锋期的特征参数在较多工况下都是无法获得的\cite{wxm2001,zxc2008}。即使在点火干扰被去除的情况下,有时候也无法很好的获得火焰前锋期的特征参数,
主要原因是火焰前锋期的持续时间较短。\par
火焰前锋期是由火焰锋面经过电极两段时产生的,其原理注定其持续时间较短。且由于缸内湍流燃烧,火焰锋面具有不确定性,同时
火焰锋面的传播过程没有很强的理论支持,不能够通过函数拟合方式很好的获得。
\begin{figure}[htb]
	\centering
	\includegraphics[width=0.8\textwidth]{thesis_figure/ion_chapter/ca10_ca90_m2s_m2e}
	\caption{\label{fig:ca10_ca90_m2s_m2e}CA10、CA90和火焰后期开始时刻和结束时刻}
\end{figure}
\subsection{火焰后期近似计算与急燃期}
通过缸压计算CA10和CA90,通过高斯拟合曲线计算火焰后期高斯拟合峰的初始位置和结束位置,得到如图\ref{fig:ca10_ca90_m2s_m2e}所示。可以看到CA10和火焰后期的初始
时刻对应,CA90和火焰后期的结束时刻对应。
\subsection{峰值相位的近似计算}
通过计算随机选取的$3000r/min$淹没工况下的连续20个循环的离子电流火焰后期峰值相位和缸压峰值相位可以得到如图\ref{fig:ionpph_pph}所示的曲线。可以看到离子电流火焰后期
峰值相位和缸压峰值相位有一定的相关性。
\begin{figure}[htb]
	\centering
	\includegraphics[width=0.8\textwidth]{thesis_figure/ion_chapter/ionpph_pph}
	\caption{\label{fig:ionpph_pph}离子电流火焰后期峰值相位和缸压峰值相位}
\end{figure}
\subsection{积分值的近似计算}
纯点火干扰的存在影响了离子电流积分值的计算,且由于点火干扰的不稳定性,其积分值也是不稳定的。
若将干扰的积分值计算在内,并不能很好的表征离子电流的真实积分值。
采用小波分析方法将点火干扰去除,再经过积分计算得到的离子电流积分值有很强的科学性。\par
通过计算随机选取的$3000r/min$淹没工况下的连续20个循环的离子电流积分值和IMEP可以得到如图\ref{fig:integer_imep}所示的曲线,可以发现离子电流和IMEP有一定的线性相关性。\par
通过分别计算随机选取的$3000r/min$淹没工况下的连续20个循环的小波分析后的离子电流积分值和双高斯曲线的积分值,可以得到如图\ref{fig:ionint_gaussint}所示的曲线。
可以发现两者之间几乎相等,说明用双高斯曲线计算的积分值可以非常好的表示离子电流积分值。\par
\begin{figure}[htb]
	\centering
	\includegraphics[width=0.8\textwidth]{thesis_figure/ion_chapter/gausint_imep}
	\caption{\label{fig:integer_imep}离子电流积分值和IMEP}
\end{figure}
因此可以用双高斯曲线计算的积分值来近似得到对应的IMEP值,而高斯曲线的面积计算有快速算法,由此在没有安装缸压传感器的发动机上可以非常快速的计算得到IMEP的值。
\begin{figure}[htb]
	\centering
	\includegraphics[width=0.8\textwidth]{thesis_figure/ion_chapter/ionint_gaussint}
	\caption{\label{fig:ionint_gaussint}离子电流积分值和高斯拟合积分值}
\end{figure}
\subsection{升高率的近似计算}
通过分别计算随机选取的$3000r/min$淹没工况下的连续20个循环的小波分析后的离子电流升高率和缸压升高率,可以得到如图\ref{fig:prr_irr}所示的曲线。可以发现
离子电流升高率和缸压升高率有一定的相关性,计算得到的相关性的值为0.78,属于高度相关。因此通过拟合计算的最大离子电流升高率估计值来估计最大缸压升高率具有一定的可行性。\par
\begin{figure}[htb]
	\centering
	\includegraphics[width=0.8\textwidth]{thesis_figure/ion_chapter/prr_irr}
	\caption{\label{fig:prr_irr}离子电流升高率和缸压升高率}
\end{figure}
通过分别计算随机选取的$3000r/min$淹没工况下的连续20个循环的小波分析后的离子电流最大升高率相位和最大缸压升高率相位,可以得到如图\ref{fig:prrph_irrph}所示的曲线。
可以发现离子电流最大升高率相位和最大缸压升高率相位有很高的相关性,经过计算的相关性的值为0.98,属于高度相关。用拟合计算得到的最大离子电流升高率相位估计值可以很好的表示最大缸压升高率相位。
\begin{figure}[htb]
	\centering
	\includegraphics[width=0.8\textwidth]{thesis_figure/ion_chapter/prrph_irrph}
	\caption{\label{fig:prrph_irrph}最大离子电流升高率相位和最大缸压升高率相位}
\end{figure}
\section{特征参数估计值的相关性和循环变动}
\subsection{相关系数和循环波动系数的计算}
循环变动系数\cite{xhbd}采用发动机中常用燃烧循环变动算法,循环变动系数的定义如下:
\begin{align}
	COV_x = \frac{\sigma_x}{\overline{x}}
\end{align}
式中$x$是任一物理量,$\sigma_x$是该物理量的标准偏差,$\overline{x}$是该物理量的平均值,其计算公式为\par
\begin{align}
	\sigma_x &= \sqrt{\sum_{i=1}^{N}(x-\overline{x})^2}\\
	\overline{x} &= \frac{\sum_{i=1}^{N}x_i}{N}
\end{align}
相关系数是研究变量之间线性关系程度的量,常见的两个变量之间的相关系数又称为简单相关系数,其定义如下:
\begin{align}
	COR_{x,y} = \frac{E(XY)-EX\cdot EY}{\sqrt{DX}\cdot\sqrt{DY}}
\end{align}
式中$X$、$Y$是两个相关的量,$EX$$DX$分别是$X$的期望和方差。\par
循环变动系数指示该特征值估计值的稳定性,相关系数指示该特征值估计值替代特征值的可靠性。选取$\SI{3000}{r/min}$淹没工况下的随机20个
循环计算双高斯曲线估计特征参数与真实信号特征参数的相关性,并计算循环变动率可以得到如表\ref{tab:parainfo}所示的表格。
而经过计算的缸压循环变动率为4.7\%,可以发现除了离子电流的积分值估计值和升高率估计值,其他的循环变动
都比缸压的循环变动率低,因此可以说明双高斯曲线拟合计算得到的特征值估计具有很好的稳定性\cite{ljw2005}。
从相关性角度来看,所有的特征值的估计值与其对应的特征值都有明显的相关性\cite{joyce2006linear}。
\begin{table}[htb]
	\centering
	\caption{\label{tab:parainfo}淹没工况下的特征参数估计值的相关性和循环变动系数}
	\begin{tabular}{|c|c|c|}
		\hline
		特征参数估计值&相关性&循环变动系数(\%)\\\hline
		火焰后期初始时刻&0.87&1.8\\\hline
		火焰后期结束时刻&0.97&1.8\\\hline
		火焰后期峰值相位&0.95&1.7\\\hline
		积分值&0.93&7.9\\\hline
		升高率&0.78&4.1\\\hline
		升高率相位&0.95&1.7\\\hline
	\end{tabular}
\end{table}
\subsection{相关性和循环波动率随转速的变化}
采用上一节的取同一工况下的随机连续20个循环来计算双高斯曲线估计特征参数与真实信号特征参数的相关性的方法,对不同转速下的估计
值的相关性进行计算,可以得到如图\ref{fig:pf_cr}和\ref{fig:irr_cr}所示。
\begin{figure}[htb]
	\centering
	\includegraphics[width=0.8\textwidth]{thesis_figure/ion_chapter/pf_cr}
	\caption{\label{fig:pf_cr}火焰后期初始相位、结束相位和峰值相位的相关性}
\end{figure}
从图\ref{fig:pf_cr}中可以看到火焰后期初始相位、结束相位和峰值相位的相关性都大于0.6,说明采用双高斯曲线计算的估计值能够近似替代真实值。且相关性随着转速的增加,
基本上趋于平稳波动状态。
\begin{figure}[htb]
	\centering
	\includegraphics[width=0.8\textwidth]{thesis_figure/ion_chapter/irr_cr}
	\caption{\label{fig:irr_cr}离子电流积分值、离子电流升高率及其相位的相关性}
\end{figure}
从图\ref{fig:irr_cr}中可以看到离子电流积分值、离子电流升高率及其相位的相关性都大于0.4,说明采用双高斯曲线计算的估计值和真实值之间有很好的线性关系。且随着转速的增加,
相关性逐渐的增大。\par
\begin{figure}[htb]
	\centering
	\includegraphics[width=0.8\textwidth]{thesis_figure/ion_chapter/pf_cov}
	\caption{\label{fig:pf_cov}火焰后期初始相位、结束相位和峰值相位的循环波动率}
\end{figure}
采用上一节的取同一工况下的随机连续20个循环来计算双高斯曲线估计特征参数与真实信号特征参数的循环波动率的方法,对不同转速下的估计
值的相关性进行计算,可以得到如图\ref{fig:pf_cov}和\ref{fig:irr_cov}所示。
\begin{figure}[htb]
	\centering
	\includegraphics[width=0.8\textwidth]{thesis_figure/ion_chapter/irr_cov}
	\caption{\label{fig:irr_cov}离子电流积分值、离子电流升高率及其相位的循环波动率}
\end{figure}
从图\ref{fig:pf_cov}中可以看到火焰后期初始相位、结束相位和峰值相位的循环波动率都小于2\%,按照平均值为360度计算误差在$\pm3$度左右。随着转速的增加,循环波动率也在不断地增加,
在$1000~r/min$至$2000~r/min$时的循环波动率为1\%以内,按照平均值为360都计算误差在$\pm2$度左右,计算精度可以满足普通的发动机控制系统计算要求。\par
从图\ref{fig:irr_cov}中可以看到离子电流积分值的循环波动较大,而离子电流升高率及其相位的循环波动率处于较小状态。积分值的循环波动较大的原因是由于拟合后的高斯曲线与实际曲线
的符合程度不一定很好,尤其是废气热电离产生的离子电流曲线并不一定完全符合高斯曲线。离子电流升高率及其相位的循环波动率较小,且随着转速的增加不怎么变化,说明该值的计算较准确。
\section{本章总结}
本章总结并分析了离子电流的基本特征参数,以及通过拟合曲线对特征参数的计算方法。\par
由于原理上看,离子电流信号不具有高频信号成分;离子电流的产生是由于燃烧产生的NO离子电离导致的,因此离子电流
的开始时刻、结束时刻和燃烧的开始时刻、结束时刻是一一对应的;且离子电流火焰后期峰值相位和缸压峰值相位有线性相关性。本章从这三个角度应证了小波分析和双高斯曲线拟合方法可以很好的对离子电流
进行拟合。\par
本章最后通过拟合曲线计算了离子电流曲线的不同特征参数,多循环角度分析了离子电流的特性和缸压特性,得出了许多有用的离子电流特性,为离子电流更深层次的应用打下了理论基础。
本章得到的主要结论如下:\par
(1)离子电流火焰后期峰值相位和缸压峰值相位的相关系数稳定在0.8以上,属于高度相关。火焰后期峰值相位估计值的循环波动率在2\%以内,且随着转速的增加相关系数基本不变,循环波动率缓慢增加。\par
(2)离子电流火焰后期的开始时刻、结束时刻和缸内燃烧的开始时刻、结束时刻的相关系数都稳定在0.6到0.8,属于中等相关。各时刻估计值的循环波动率都在2\%以内,且随着转速的增加相关系数
基本不变,循环波动率缓慢增加。\par
(3)双高斯拟合曲线的积分值与离子电流积分值基本相等,且和IMEP的相关系数稳定在0.9左右,属于高度相关,且随着转速的变化基本不变。离子电流积分值的估计值的循环波动率较大,说明积分值的估计值计算有一定的误差。\par
(4)缸压最大压力升高率和离子电流火焰后期最大升高率的相关系数在0.4以上,且随着转速的增加可以达到0.8。缸压最大压力升高率相位和离子电流火焰后期最大升高率的相关系数在0.6以上,且随着转速的增加可以达到0.9。
这两个估计值的循环波动率都在5\%以内。\par
(5)离子电流积分值和升高率估计值有一定波动,其他特征值的估计值都具有很好的稳定性。