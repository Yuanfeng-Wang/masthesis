\chapter{总结及展望}
\section{全文总结}
目前,国内外对离子电流基本特性的研究很多,电容式离子电流检测电路具有性价比高,安装方便的特点,但是由于没有屏蔽点火干扰,随着转速的增加,容易造成点火干扰淹没正常的离子电流信号的弊端。
本文通过小波分析的方法很好的将离子电流信号中的点火干扰信号分离出去,获得了理想的离子电流信号。同时比较不同的小波基函数以及小波分析层级,得出了最佳的小波基函数和分解层数。
同时采用了双高斯拟合算法,能够快速地计算出离子电流的各种特征值,能够大大缩小需要的发动机控制系统的计算资源;比较了特征值的估计值和真实值之间的相关性和循环波动率,验证了
双高斯曲线拟合算法估算离子电流特征值的有效性。\par
本文通过小波分析方法和双高斯曲线拟合方法对离子电流的火焰后期峰值、火焰后期时间、离子电流积分值和离子电流升高率等进行了分析,得到了如下的结论:\par
(1)点火干扰信号是由点火线圈产生的,具有稳定的频率特性和振幅特性。离子电流是由于燃烧产生的离子在电场的作用下的移动产生的,不具有震荡信号成分。因此从采集到的离子电流信号中分离出
具有震荡特性的点火干扰信号具有坚定的理论基础。\par
(2)小波分析方法通过小波基函数逐层对信号进行对比分离出近似信号和细节信号,能够很好地提取出信号中的震荡成分。采集信号是离散信号,需要使用离散小波基。比较四种常用的离散小波基和十层小波分解发现
采用dmey小波基函数和第五层近似分解可以很好地去除点火干扰信号,提取出真实的离子电流信号。\par
(3)根据离子电流的理论原理,火焰后期由NO离子的热电离导致的信号近似高斯函数,可以用高斯曲线近似分析。而由于排气过程不能将缸内的所有废弃排除缸内,断油循环和断火循环中仍然存在部分带NO离子
的废弃,仍然可以检测到离子电流。同样道理,在正常循环的点火之前由于存在部分废气,离子电流检测电路可以检测到轻微的离子电流信号。这部分废弃导致的离子电流信号也可以用高斯函数近似分析。
由此整个检测到的离子电流信号可以用两个高斯曲线拟合来进行分析。\par
(4)利用双高斯曲线拟合得到的离子电流特征参数估计值和缸压特征值有相关性。离子电流的火焰后期峰值相位估计值和缸压峰值相位明显的相关性;离子电流的火焰后期开始时刻估
计值、结束时刻估计值和燃烧放热率CA10、CA90有一定的相关性;离子电流积分值估计值和平均指示压力有明显的相关性;最大离子电流升高率估计值和最大缸压升高率有一定的相关性;
最大离子电流升高率对应曲轴转角估计值和最大缸压升高率对应曲轴转角有明显的相关性。\par
(5)离子电流积分值和升高率估计值有一定波动,其他特征值的估计值都具有很好的稳定性。
\section{工作展望}
本文的实验和拟合计算方面还有很多地方可以提高,主要有:\par
(1)电磁屏蔽\cite{cyb2012,tb2009}做得不够细致导致采集信号的噪声信号过大,导致通过差值计算得到的纯点火干扰信号夹杂噪声,会在一定程度上影响小波分析提取真实离子电流信号的效果;\par
(2)废气产生的热电离信号可以用其他函数来进行近似计算;\par
(3)两个点火干扰信号起始位置计算的算法需要优化,对真实信号和纯点火干扰信号做小波分析后相减的算法对点火干扰信号的起始位置计算要求很高;