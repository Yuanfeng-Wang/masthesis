\chapter{绪论}
\section{前言}
早在十九世纪,人们就观察到碳氢化合物在燃烧时会产生自由离子,使得火焰和燃烧产物具有导电能力。这一现象被注意到,
并逐渐运用到燃烧过程研究工作。在火花塞点火式发动机中,当燃烧发生时,火花塞电极之间就产生大量的离子。如果在火花塞
两极之间加上适当的直流偏置电压就会形成离子电流,通过该电流可以对发动机进行检测实现对发动机的实时监测。但是,由于
以往对发动机控制技术要求不高,同时受发动机控制软、硬件系统的整体技术水平限制,该方法并未引起人们的广泛
注意。近年来,随着对发动机效率及排放性能要求的逐渐提高,该方法逐渐引起重视。离子电流检测技术的最
初应用主要是面向于失火现象的检测,在美国国家环保局(EPA)第II代车载诊断装置(OBDII)法规颁
布后,要求在所有负荷和转速下实现100\%失火检测\cite{grzys}。世界上许多厂家采用的是曲轴转速波动法,但这种方法在
路况较差的情况下会产生干扰信号,影响检测结果,而采用离子电流法则可以避免上述问题,实现100%失火检测。由此,对离子
电流检测技术的研究逐渐由对燃烧过程中的失火、爆震等现象的定性检测,逐渐向空燃比、燃烧相位及缸内燃烧压力状态等参数的定量检测上来。
\section{国内外研究现状}
目前,对离子电流检测技术的研究主要集中于离子电流的形成机理、应用途径、特征提取及参数估计方法以及实际控制应用等四个方面。
\subsection{离子电流的形成机理研究}
早在1953年Winch和Mayes\cite{winch1953method}就利用了偏置电压式的离子电流检测电路记录了离子电流信号,并以此分析了缸内燃烧状况。在20世纪
50年代,出现了大量的研究火焰中化学离子平衡和火焰后燃区的研究成果。进入80年代,汽车领域的科技工作者致力于离子电流信号的研究,取
得了很多重要的成果。1986至1997年Nick Collings\cite{collings1986knock,collings1991plug,collings1985knock}发表了利用火花塞作为传感器检测爆震的三篇论文,介绍了离子电流
法的基本原理和离子电流的发展历程,并完善了爆震检测的方法。1996年Andre Saitzkoff\cite{saitzkoff1996ionization}发表了火花塞离子电流的平衡计算的论
文,利用偏置电源式离子电流检测电路测量了离子电流信号,通过分析得到了火花塞离子电流的近似理论计算公式。2000年Wayne University
的D. Schneider\cite{schneider2000experimental}发表了离子电流与内燃机工作参数之间关系的博士论文。

在离子电流的形成机理研究方面, Saitzkoff和Reinmann等人\cite{saitzkoff1996ionization}率先对$N_{2}$,$H_{2}O$,$CHO$和$NO$四种基团在离子电流形成过程中所起的作用
进行了对比分析。研究认为燃烧火焰电离后能够定向移动的自由离子只占极小部分,其原因在于移动能力上的差异。同时,研究表明占离子总浓
度1%的$NO$基团产生离子电流的能力最强。Kessler等人\cite{kessler}在柴油机上采用光学方法对离子电流的研究表明,由于柴油和汽油在燃烧时产
生自由离子的化学反应过程不同,二者产生的离子电流信号波形差异显著。在汽油机上形成的离子电流中,电子是负电荷的主要载体,而对于
柴油机上的所形成的离子电流,带负电的离子也需考虑。A.Franke\cite{franke2003analysis}在对汽油机离子电流产生的物质来源的分析基础上,首次将离子
电流的产生划分为火焰前锋期和火焰后期两个阶段,认为在火焰前峰期的离子电流主要是由火焰前锋面与火花塞电极接触时所产生的,而
在火焰后期热电离过程是形成离子电流的主要原因。

除了离子电流产生的化学动力学机理外,气缸结构及火花塞位置等物理因素也是影响离子电流波形的重要成因。L.Peron等人\cite{peron2000limitations}对离子电流
信号的局限性分析认为,离子电流生成机理并不是决定信号波形的唯一原因。
\subsection{基于离子电流的检测应用研究}
目前对离子电流的应用已经从初期对失火及爆震的定性研究转入对燃烧参数的定量研究。在空燃比检测方面,Reinmann\cite{reinmann1997local}利用离子电
流公式理论和实验论证拟合了空燃比-离子电流公式。Kenneth Ratton和Ming C. Lai等人\cite{balles1998cylinder}利用火花间隙电离检测研究了缸
内空燃比的近似值,观察到了空燃比和离子电流特征之间的较强相互关系。其研究认为,从整体性能上看,离子电流检测无法取代缸压检测,二者
信号质量和性质都完全不同。但通过具体工况下对离子电流波形的具体分析,该方法仍具备有对空燃比进行实时反馈的潜力。在对缸内燃烧
压力状态的检测方面,Chao.F等人\cite{zhu2003mbt}提出,部分负荷工况下,对离子电流信号进行了多循环平均处理,通过处理后信号的第二峰位置
及幅值,实现了对缸内燃烧峰值压力及其位置的准确预测。
\subsection{特征提取及参数估计算法}
在基于离子电流的燃烧特征提取算法方面,J.Forster等人\cite{forster1999ion}率先基于离子电流的频域特征,实现了对爆震及其强度的准确估计。该研
究展示了爆震强度与离子电流频域信号高频区域内幅值的相关性。此后,一系列针对产品发动机爆震检测的相关研究也相继出现。这些研究
中普遍采用了离子电流的频域特征实现了较高质量的爆震检测。

此外,Gerard等人\cite{malaczynski2003real}针对不同燃烧边界条件下,离子电流信号变化规律复杂的情况,提出了基于信号时域特征,采用主成分分析方
法对离子电流信号进行了特征提取,该方法可以有效的对离子电流信号观测窗内观测样本进行降维,从而减小后续的参数估计算法的计
算量。

在基于离子电流的参数估计方法方面,由于离子电流信号的产生机理复杂,信号波形变化规律多样,在利用该信号进行燃烧特征检测时
,大多采用了数学模型来建立电流信号特征与被估参数之间的映射关系。在这方面,Gazis等人\cite{panousakis2006ion}通过双高斯曲线拟合的方法,对电流
信号的时域特征进行了提取,并针对离子电流信号的非线性变化特征,采用神经网络方法实现了对汽油机燃烧峰值压力位置的准确预测。
\subsection{基于离子电流反馈的实时控制}
Magnus等人\cite{glavmo1999closed}在研究柴油机EGR和颗粒污染对离子电流的影响时,尝试了在稳态工况下,通过多循环均值后的离子电流信特征值
号对燃烧始点进行反馈和控制。先预定一个离子电流信号强度阈值,当检测到的离子电流信号首次越过该阈值时,该时刻被定义为燃烧
始点。对燃烧始点检测成功后,可以通过调整燃油十六烷值和空气、燃油以及发动机温度等相应方法,对滞燃期加以调整。
\subsection{国内研究现状}
在国内许多科研院校和研究所针对离子电流也展开了大量的研究工作,其中典型的诸如,同济大学李理光课题组针对汽油机的点燃方式和
均质压燃方式下离子电流的产生机理及信号特征变化规律进行了研究,分析了两种燃烧模式下信号的产生机理及影响因素。在此基础上,研
究了基于该信号的特征提取及燃烧信息反馈方法。进而,针对现有发动机传感及控制方法的不足,提出了以离子电流信号为反馈量,在发
动机上实现基于循环的燃烧信息反馈及控制思想。西安交通大学的吴筱敏\cite{gh2010}和天津大学的谢辉等课题组也基于离子电流检测手段进行汽
油机爆震、空燃比检测和燃烧相位判断等相关研究。
\section{本课题的意义和主要研究内容}
常见的离子电流检测电路分为电容式和偏置电源式。偏置电源式检测电路可以将点火干扰屏蔽,获得的离子电流曲线便于分析计算\cite{cyb2012},但
是由于需要使用高压硅堆和外置电源成本太高。电容式检测电路结构简单,成本低廉,但是点火干扰阶段容易和火焰前锋期甚至包括焰后期重
叠,导致离子电流的一些特征参数无法识别。如果能够提出一种较好的信号处理方法处理点火干扰阶段信号,将会使成本低廉、结构简单的电容
式检测电路得到推广。
\par正常情况下,采用电容式离子电流检测电路得到的离子电流信号曲线可以分为三个时期;火花塞放电期,火焰前锋期,火焰后期。火花塞放
电期间,由于次级线圈放电导致检测电路中产生了较大的电流。但如果火花塞附近发生早燃,则在点火信号之前或之后将产生一些峰值\cite{eriksson1997closed}。如
果火花塞点火放电时间过长,将会和火焰前锋期甚至是火焰后期的离子电流信号重叠,从而影响了离子电流信号的分析。
\par在重叠情况下,仍然可以预估甚至提取离子电流。由于点火干扰的根本原因是点火能量在点火线圈中的震荡,其频率是有一定规律的。火焰前锋
期的电流信号主要和火焰传播过程相关,火焰后期的电流信号和缸内的热力学过程相关,这两个时期的电流信号频率和点火线圈中的能量震
荡频率区分明显。这是离子电流被淹没而仍然可以预估甚至提取的基本原理。所以为了能够准确地提取离子电流信号,首先需要通过对电容式
离子电流检测电路和纯点火干扰信号进行分析,以便了解点火干扰信号的频幅特性。而在失火情况下,由于燃烧没有发生,离子电流信号没有产生。所以
失火是检测电路测定的信号就是纯点火干扰信号。除了信号提取的方法之外,Saitzkoff\cite{saitzkoff1996ionization}做了很多研究,提出了近似的离子电流计算公式。通过将
这些公式进行简化提出一些参数函数,也能够很好的预测离子电流的波形和特征参数,甚至可以达到预测空燃比的要求。
\par离子电流的火焰前锋期信号是由于燃烧的火焰前锋造成的,所以该时期的离子电流能够很好的用于分析缸内的燃烧状况,比如计算燃烧锋面的速度。而
离子电流的火焰后期是由于火焰前锋面已经离开火花塞电极,但是缸内燃烧仍在继续,导致的缸内热力学变化引起的热电离而形成的,所以能够很好的反应内燃机的负荷,缸内压力变化等。
\par所以正确地获得离子电流两个时期的信号曲线对于分析缸内的燃烧情况是非常重要的。但是这两个时期的信号极容易被点火干扰信号淹没,导致无法得到
准确的离子电流信号。通过上述的一些方法去除干扰信号,就可以帮助更好的认识离子电流信号,为后续的研究扫清障碍。即使无法完整准确地消除点火放
电干扰信号,仍然能够通过上述方法统一处理每个循环的离子电流信号,得到该处理方法下的某个稳定的特征变量,从而指导反馈控制发动机的运行参数。
